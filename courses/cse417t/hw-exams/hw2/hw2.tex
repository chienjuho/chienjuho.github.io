\documentclass[11pt]{article}
\setlength{\topmargin}{-0.5in}
\setlength{\textwidth}{6.5in}
\setlength{\oddsidemargin}{0in}
\setlength{\textheight}{9in}

%\usepackage{multirow}
%\usepackage{rotating}
\usepackage{bm}
\usepackage[fleqn]{amsmath}
\usepackage{natbib}
\usepackage{palatino}
%\usepackage{url}
\usepackage{hyperref}

\begin{document}

\title{CSE 417T: Homework 2}
\date{Due: October 7 (Friday), 2022}

\maketitle

\noindent \textbf{Notes: } 
\begin{itemize}

\item Please submit your homework via Gradescope and check the \href{http://chienjuho.com/courses/cse417t/hw_instructions.html}{\underline{submission instructions}}.

\item There will be two submission links for homework 2: one for report and the other for code. \textbf{Your score will be based on the report.} The code you submit is only used for checking the correctness and for running plagiarism checkers. You might lose points if there are issues with your code.

\item Make sure you \textbf{specify the pages for each problem correctly}. You \textbf{will not get points} for problems that are not correctly connected to the corresponding pages.

\item Homework is due \textbf{by 11:59 PM on the due date.} Remember that
  you may not use more than 2 late days on any one homework, and you
  only have a budget of 5 in total.

\item Please keep in mind the collaboration policy as specified in the
  course syllabus. If you discuss questions with 
others you \textbf{must} write their names on your submission, and if
you use any outside resources you \textbf{must} reference
them. \textbf{Do not look at each others' writeups, including code.}

\item There are 5 problems on 2 pages in this homework. 

\end{itemize}

\noindent \textbf{Problems:}

\begin{enumerate}
\item (50 points) 
    Please download the following files: one python file and two data files\\
    \url{http://chienjuho.com/courses/cse417t/hw2/hw2_files.html}


  Complete \texttt{hw2.py} and submit it to Gradescope. 
  We have provided two function headers (\texttt{find\_classification\_error} and \texttt{logistic\_reg}) that you should complete.
  You can write additional functions and additional code in the main function to help you complete the experiments below.
  
  If you are interested, you can read more about the ``Cleveland'' dataset we'll be using here:\\
    \url{https://archive.ics.uci.edu/ml/datasets/Heart+Disease}

  Descriptions:\\
  Implement and learn a logistic regression model on the data in \texttt{clevelandtrain.csv}.
  Note that the labels (\texttt{heartdisease::category|0|1}) in the data set are $0/1$ so you'll need to convert those to $-1/+1$ to ensure that everything
  we've done in class is still valid. 
  Use a learning rate of $\eta = 10^{-5}$ and automatically terminate if the magnitude of every element of the gradient is less than $10^{-3}$.
  Initialize the weight vector to a vector of all zeros. 
  Train the model three times with a different bound on the maximum number of iterations each time: $10^4$, $10^5$, and $10^6$.
   (The termination condition based on the magnitude of the gradient still applies). 
   Use each model to classify the data using a cutoff probability of 0.5.
    
   In your report submission:
   \begin{itemize}
    \item[a]
    For each of the maximum number of iterations, 
    report $E_{\text{in}}$ (the cross-entropy error) on the training data set, the binary classification error on
    both the training and test data sets, and how long the training process took (in seconds).
    What can you say about the generalization properties of the logistic regression model? 
    How does increasing the maximum number of iterations affect the model's performance?

%    \item[b]
%    Now train and test a logistic regression model using the
%    inbuilt matlab function \texttt{glmfit} (learn about and use the ``binomial'' option, and check the label format). 
%    Report $E_{\text{in}}$ (the cross-entropy error), the binary classification error on
%    both the training and test data sets, and how long the training process took (in seconds).
%    (Note that \texttt{glmfit} returns a weight vector, so you might need to calculate $E_{\text{in}}$ yourself.)

    \item[b]
    %Now get back to your own implementation. 
    We now normalize the features, i.e., 
    scale each feature by subtracting the mean and dividing by the standard deviation for
    each of the features in advance of calling the learning algorithm. 
    Describe how you perform the normalization for the training set and testing set (i.e., what's the mean and the variance that you use for normalization).
    Remember the two conditions: 1) training set and test set need to be from the same distribution, and 2) information of test set cannot be used in training.

    Experiment with the learning rate $\eta=0.01, 0.1, 1, 4, 7, 7.5, 7.6, 7.7$.
    For the termination criteria, set the maximum number of iteration to $10^6$. Terminate when the magnitude of every element of the gradient is less than $10^{-6}$.
    For each learning rate, 
    report $E_{\text{in}}$ (the cross-entropy error) on the training data set, the binary classification error on
    both the training and test data sets, \textbf{the number of iterations required to terminate}, 
    and how long the training process took (in seconds).
    Did normalizing the data affect the performance of the model? 
    What is the effect of changing learning rate $\eta$?
   \end{itemize}


Additional note:
\begin{itemize}
    \item You can present the numbers in a table, with each row being a method (e.g., your implementation with different max \# iterations, etc) and each column being a measure ($E_{\text{in}}$, etc) to make it clear to read.
    \item While you should feel free to try out stochastic gradient descent or mini-batch algorithms. 
In your submission to homework 2, please report your results using gradient descent algorithm (non-stochastic version). 
    \item You can try standard libraries (e.g., \texttt{sklearn.linear\_model.LogisticRegression}) to see whether your implementation outputs reasonable outcomes.

\end{itemize}

\item (10 points) LFD Problem 2.22

\item (15 points) LFD Problem 2.24. You do not need to submit your code for this problem.

%\item (10 points) LFD \textbf{Exercise} 3.4 (a) to (c). Note that the definition of $H$ is in Equation 3.6 in Page 87 of LFD. You can assume the property in LFD Exercise 3.3 as given.

\item (15 points) LFD Problem 3.4. For part (c), you will need to read the description of LFD Problem 1.5, but you do not need to complete LFD Problem 1.5.

\item (10 points) LFD Problem 3.19. Discuss in each case whether learning is feasible (i.e., when $N$ goes large, does the generalization error become smaller) and explain why. 

Below are some notation clarifications in the problem.
\begin{itemize}
    \item We use $D=\{(\vec{x}_1,y_1),...,(\vec{x}_N,y_N)\}$ to denote the training dataset. When you see $\vec{x}_n$ (or $\bm{x}_n$ in the textbook), it usually implies that it's the n-th point in the training dataset.
    \item In (b), $\phi_n(\vec{x})$ is the n-th element of $\Phi(\vec{x})$, i.e., $\Phi(\vec{x})=(\phi_1(\vec{x}),\phi_2(\vec{x}),...)$.
    \item In (c), $(i,j)$ is the 2-d vector with the first element being $i$ and second being $j$.
\end{itemize}




\end{enumerate}


\end{document}
