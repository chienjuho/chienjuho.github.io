\documentclass[11pt]{article}
\setlength{\topmargin}{-0.5in}
\setlength{\textwidth}{6.5in}
\setlength{\oddsidemargin}{0in}
\setlength{\textheight}{9in}

%\usepackage{multirow}
%\usepackage{rotating}
\usepackage[fleqn]{amsmath}
\usepackage{natbib}
\usepackage{palatino}
\usepackage{url}
\usepackage{hyperref}

\begin{document}

\title{CSE 417T: Homework 4}
\date{Due: November 14 (Monday), 2022}

\maketitle


\noindent \textbf{Notes: } 
\begin{itemize}

\item Please submit your homework via Gradescope and check the \href{http://chienjuho.com/courses/cse417t/hw_instructions.html}{\underline{submission instructions}}.

\item Please download the following files for this homework.\\
    \url{http://chienjuho.com/courses/cse417t/hw4/hw4.html}

\item Homework is due \textbf{by 11:59 PM on the due date.} Remember that
  you may not use more than 2 late days on this homework, and you
  only have a budget of 5 in total.

\item Please keep in mind the collaboration policy as specified in the
  course syllabus. If you discuss questions with 
others you \textbf{must} write their names on your submission, and if
you use any outside resources you \textbf{must} reference
them. \textbf{Do not look at each others' writeups, including code.}

\item Please comment your code properly.

\item There are 5 problems on 3 pages in this homework. 

\end{itemize}

\noindent \textbf{Problems:}

\begin{enumerate}

\item (50 points) The purpose of this problem is to write code for bagging decision trees and computing the out-of-bag error.
You may use \texttt{sklearn.tree.DecisionTreeClassifier} function\footnote{https://scikit-learn.org/stable/modules/generated/sklearn.tree.DecisionTreeClassifier.html}, which learns decision trees (read the documentation carefully), but do not use the inbuilt functions for producing bagged ensembles. 
%Note that it only returns the out-of-bag error.
%You may use matlab's inbuilt \texttt{fitctree} function, which learns decision trees using the CART algorithm (read the documentation carefully), but do not use the inbuilt functions for producing bagged ensembles. 
%In order to do this, you should complete the stub \texttt{BaggedTrees} function. Note that it only returns the out-of-bag error. You may want to use other functions that actually construct and maintain the ensemble. 
You may assume that all the \textbf{x} vectors in the input are vectors of real numbers, and there are no categorical variables/features. You will compare the performance of the bagging method with plain decision trees on the handwritten digit recognition problem (the dataset is in \texttt{zip.train} and \texttt{zip.test},
available from \url{http://amlbook.com/support.html}.~\footnote{Check the links to ``training set'' and ``test set''.})

We will focus on two specific binary classification problems -- distinguishing between the digit one and the digit three, and distinguishing between the digit three and the digit five. You need to report the results for both problems (1 versus 3 and 3 versus 5).

\begin{itemize}
    \item Code (Complete and submit \texttt{hw4.py})
    \begin{itemize}
        \item[a] Write code that creates training and testing datasets from \texttt{zip.train} and \texttt{zip.test} 
        for the two binary classification problems. 
        The first will only include digits classified as one or three. 
        The second will only include digits classified as three or five. 
        
        \item[b] Complete the implementation of \texttt{bagged\_trees}. 
        For each bag, please learn a fully grown tree and use information gain as the split criterion.
        You do not need to implement the random feature split.
        You can update the function headers or write new functions for your convenience. 
        The requirement is that you need to be able to calculate and plot the out-of-bag error as a function of the number of bags from 1 to the number specified as input (\texttt{num\_bags}), 

        \item[c] Learn a single decision tree model (again, fully grown with information gain as the split criterion) from the training dataset and calculate the test error on the test set.
    \end{itemize}
        %Run the provided \texttt{OneThreeFive} script, which creates training datasets based on the one-vs-three and three-vs-five cases we are interested in, and calls both the in-built decision tree routine and your bagging code, printing out the cross-validation error for decision trees and the OOB error for your bagging implementation. Report the results in your writeup.
        %\item[c]  Now, learn a single decision tree model for each of the two specified problems (one-vs-three and three-vs-five) on the training data, and test their performance on \texttt{zip.test} – what is the test error? Similarly, learn a single ensemble of 200 trees on the training data for each of the two specified problems and test the performance of the ensembles on the test data. Report your results.
    
    \item Report:\\
        For each of the problems (1 versus 3 and 3 versus 5):
        \begin{itemize}
            \item[a] Plot the OOB error for bagging decision trees with the number of bags from 1 to 200 (with x-axis being the number of bags, and y-axix being the OOB error). Make sure the axes are clearly labeled.
            \item[b] Report the OOB error of bagging decision trees (with 200 trees) and the test errors of (1) a single decision tree and (2) bagging decision trees (with 200 trees).
        \end{itemize}
        
        Summarize and interpret your results in one or two concise paragraphs as part of your writeup.
        You should at least comment on 1) the differences between the one-vs-three and three-vs- five problems, 2) the effect of increasing the number of bags, and 3) the connection between OOB error and test error.
\end{itemize}

%\item (50 points) Implement AdaBoost using decision stumps learned using information gain as the weak learners (you may use the \texttt{fitctree} function to implement the weak learner. Look at the "deviance" split criterion), and apply this to one-vs-three and three-vs-five problems (as described in Question 1) on the \texttt{zip.train} and \texttt{zip.test} data. In order to do this, you should complete the stub \texttt{AdaBoost} function. Graphically report the training set error and the test set error as a function of the number of weak hypotheses, and summarize and interpret your results.

\item (20 points)
You have been hired by a biologist to learn a decision tree to determine whether a mushroom is poisonous. You have been given the following data:

\vspace{-7pt}
\begin{center}
\begin{tabular}{| c | c | c | c |}
    \hline
    Color & Stripes & Texture & Poisonous? \\
    \hline
    Purple  & No    & Smooth    & No \\
    Purple  & No    & Rough     & No \\
    Red     & Yes   & Smooth    & No \\
    Purple  & Yes   & Rough     & Yes \\
    Purple  & Yes   & Smooth    & Yes \\
    \hline
\end{tabular}
\end{center}
\vspace{-7pt}

Use ID3 to learn a decision tree from the data (this is a written exercise -- no need to code it up):
\begin{itemize}
    \item[(a)] What is the root attribute of the tree? Show the computations.
    \item[(b)] Draw the decision tree obtained using ID3.
\end{itemize}

\item (10 points)
Think about weak learners in AdaBoost for a 2-class classification problem.
Suppose you're using depth 0 decision trees, which
  simply return the weighted majority class of the data points as the
  classification, as the weak learner. 
  Imagine that, at the first iteration, 80\% of the data points were positive 
  and 20\% of the data points were negative.
  What would the cumulative weight of positive points and the cumulative weight of negative data points be after one round of boosting)?
  From your result, do you think whether using
  depth 0 decision trees as weak learners is a good idea?

\item (5 points)
Assume we want to use bagging to solve a regression problem. 
Argue why using linear regression as weak learners for bagging might be a bad idea.
Hint: Think about what the space of the hypothesis set is after bagging.

%\item (5 points)
%Suppose your input data consists of the following $(x, y)$ pairs:
%\[
%(3, 5); (5, 6); (7, 9); (2, 11); (3, 8)
%\]
%What value of $y$ would you predict for a test example where $x = 3.2$ 
%using the 3-nearest neighbors regression?

\item (15 points) Read the article(s) in one of the topics below:\\
\url{https://docs.google.com/document/d/1QmKDwodF9gTZQzrzQuES6EXDDwmqNWOfRFqGqTP83k8/edit?usp=sharing}
%\url{https://wustl.box.com/s/x68vmeajbgeoz1owsz640i8tvjr1ezxs}.

Answer the following questions:
\begin{itemize}
    \item[(a)] Summarize the article(s) in one paragraph.
    \item[(b)] Rephrase the issues raised in the article using the language you learned in this course.
    \item[(c)] Propose potential approaches to mitigate the issues raised in the article.
\end{itemize}

This question will be graded in a loose manner.
The grading will focus on 
1) whether you have put thoughts into your answer, and
2) whether your answers are logical (i.e., it's okay to give unpopular opinion, but make sure you provide your reasoning).

Example grades (not an exhaustive list, but just an illustration):\\
You will get full points if the summary is accurate, and the proposed approach makes sense (e.g., feasible if you are given the data and resource). 
You will get 10 points with an accurate summary but non-negligible flaws in the proposed approach.  
You will get 5 points for providing inaccurate summary (e.g., incorrect mapping to the language in this course).

\end{enumerate}

%\noindent
%For code submissions, please submit all files that will be needed to return the results. You are free to modify \texttt{OneThreeFive.m} and/or include additional files. The only requirements are (1) the results can be obtained by running \texttt{OneThreeFive.m} and (2) the two stub files are filled in.  
%It is okay to have additional files. 
%It is also okay if you want to modify the input/output specification slightly, but please make sure you commented it appropriately.

\end{document}
