\documentclass[10pt]{article}
\setlength{\topmargin}{-0.5in}
\setlength{\textwidth}{6.5in}
\setlength{\oddsidemargin}{0in}
\setlength{\textheight}{9in}

\newcommand{\wlin}{\mathbf{w}_{\text{lin}}}
\newcommand{\out}{\text{out}}
\newcommand{\val}{\text{val}}

%\usepackage{multirow}
%\usepackage{rotating}
\usepackage[fleqn]{amsmath}
\usepackage{natbib}
\usepackage{palatino}
\usepackage{url}
\usepackage{graphicx}

\usepackage[fleqn]{amsmath}
\usepackage{amsfonts}
\usepackage{mathtools}
\usepackage{natbib}
\usepackage{palatino}
\usepackage{url}
\usepackage{graphicx}

\DeclarePairedDelimiter\floor{\lfloor}{\rfloor}

\newcommand{\vecw}{{\mathbf{w}}}
\newcommand{\vecxn}{{\mathbf{x}_n}}

\begin{document}

\title{CSE 417T: Homework 2 Solution Sketches}



\maketitle

\noindent \textbf{Note:} These are not intended to be comprehensive,
just to help you see what the answers should be.

\begin{enumerate}

\item[Implementation] $E_{\text{in}}$ and the classification error for the
  non-normalized data are reported below:

  \begin{center}
  \begin{tabular}{l|c|c|c}
    Algorithm & $E_{\text{in}}$ & \multicolumn{2}{c}{Classification
      Error} \\
    & & Training & Testing \\ \hline
%    \texttt{glmfit} & .4074 & .1711 & .1103\\
    GD 10000 & .5847 & .3092 & .3172 \\
    GD 100000 & .4937 & .2237 & .2069 \\
    GD 1000000 & .4354 & .1513 & .1310 \\
  \end{tabular}
  \end{center}

  Main takeaways: the generalization performance is excellent, in the
  sense that testing error is very close to training error, so the
  model is definitely not overfitting. The performance improves
  significantly with lower $E_{\text{in}}$, which makes sense given
  that the model generalizes well. Therefore, the benefits of
  optimizing the model for the training set logistic error measure are
  good. 
  %\texttt{glmfit} (with the \texttt{'binomial'} option for
  %logistic regression) is much faster, and also better in terms of
  %performance (which again makes sense given the generalization is
  %good here). Once we get to a million iterations, gradient descent is
  %getting close to the performance of the optimizer in matlab.

  For the normalization, note that we need to normalize the training 
  and test sets using the mean/variance of the training set.
  %the only change you need to make is to use
  %something like\\ 
  %\texttt{Xtrain = clevelandtrain(:,1:13); Xtrain = zscore(Xtrain);}\\
  %before calling your \texttt{logistic\_reg} function. 
  It achieves the
  same error ($E_{\text{in}}$) much faster with appropriately chosen $\eta$.
  %as \texttt{glmfit} very fast.
  %in fact,
  %in time comparable to \texttt{glmfit} for some settings of
  %$\eta$. Here are some examples of the number of iterations needed
  %(doesn't terminate by 10000 much beyond 7).

  \begin{center}
  \begin{tabular}{l|c|c|c|r}
    $\eta$ & $E_{\text{in}}$ & Binary training error & Binary test error & \# iterations \\
    \hline
    0.01 & 0.4074 & 0.1711 & 0.1103 & 23222 \\
    0.1 & 0.4074 & 0.1711 & 0.1103 & 2319 \\
    1 & 0.4074 & 0.1711 & 0.1103 & 229 \\
    4 & 0.4074 & 0.1711 & 0.1103 & 53 \\
%    6 & 0.4074 & 0.1711 & 0.1103 & 32 \\
    7 & 0.4074 & 0.1711 & 0.1103 & 46 \\
    7.5 & 0.4074 & 0.1711 & 0.1103 & 182 \\
    7.6 & 0.4074 & 0.1711 & 0.1103 & 453 \\
    7.7 & 0.4083 & 0.1645 & 0.1172 & 1000000 \\
    
  \end{tabular}
  \end{center}

    \item[Problem 2.22] First, compute $E_{\text{out}} (g^{\mathcal{D}}) =
      \mathbb{E}_{\mathbf{x}, y} [ (g^{\mathcal{D}}(\mathbf{x}) - y)^2
      ]$. Replacing $y$ with $f(\mathbf{x}) + \epsilon$ and expanding,
      we get $\mathbb{E}_{\mathbf{x}, \epsilon}
      (g^{\mathcal{D}}(\mathbf{x}) - f(\mathbf{x}))^2 - 2
      (g^{\mathcal{D}}(\mathbf{x}) - f(\mathbf{x})) \epsilon +
      \epsilon^2$. Since $\epsilon$ is independent of $\mathbf{x}$ the
      second term disappears, and we get $\mathbb{E}_{\mathbf{x}}
      (g^{\mathcal{D}}(\mathbf{x}) - f(\mathbf{x}))^2 +
      \mathbb{E}_{\epsilon} [\epsilon^2]$. The second term is
      $\sigma^2$ and when taking the expectation over the first one we
      get back the bias + variance.

    \item[Problem 2.24] 
      \begin{itemize}
        \item[(a)] When you fit the two data points to a line, you get
          a slope of $x_1 + x_2$ and an intercept of $-x_1 x_2$. Since
          the expectations of both of these terms are 0, you get
          $\bar{g} (x) = 0$.

        \item[(b)] Generate many datasets, compute $a, b$ for
        each of them, and essentially use the average of a large number of samples to simulate expectation of the quantities of interests.
        Compute $\bar{g} (x) = \bar{a} x +
        \bar{b}$. Now,
        \[ E_{\text{out}} (a, b) = \frac{1}{2} \int_{-1}^{1} \,dx (ax
        + b - x^2)^2 = \frac{1}{5} + \frac{a^2 - 2b}{3} + b^2 \]
        The bias can be computed as $E_{\text{out}} (\bar{a},
        \bar{b})$. The variance is the variance of $ax+b$, which is
        given by $x^2 \text{Var} (a) + 2x \text{Cov}(a,b) + \text{Var}
        (b)$. Integrating over $x$ from $-1$ to $1$, the second term
        disappears, and we are left with $\frac{1}{3} \text{Var} (a) + \text{Var}
        (b)$ which we can use for the calculation.

        \item[(c)] Run simulations and report the results, which
          should be similar to the analytical ones below.

        \item[(d)] Analytically, $\bar{a} = \bar{b} = 0$. Now in order
          to use the formulae above, we still need to compute
          $\bar{a^2} = \mathbb{E} [(x_1 + x_2)^2] = 2/3$ and
          $\bar{b^2} =\mathbb{E} [x_1^2 x_2^2] = 1/9$. Plugging into
          the formula for $E_{\text{out}}$ above, we get $8/15$, with bias
          and variance (also computed as above) of $1/5$ and $1/3$
          respectively. 

        \end{itemize} 


\item[Problem 3.4]
\begin{itemize}
  \item[a] Consider two cases. If $y_n \mathbf{w}^T \mathbf{x}_n < 1$
    then $E_{n} (\mathbf{w}) = (1 - y_n \mathbf{w}^T \mathbf{x}_n)^2$
    which is differentiable. When $y_n \mathbf{w}^T \mathbf{x}_n > 1$,
    $E_{n} (\mathbf{w}) = 0$, which is differentiable. Therefore, the
    only case where a problem could occur would be when $y_n
    \mathbf{w}^T \mathbf{x}_n < 1$. From the left, the gradient is
    $-2(1 - y_n \mathbf{w}^T \mathbf{x}_n) y_n \mathbf{x}_n$ which
    approaches 0, and from the right the gradient is 0, so the
    gradient is continuous, hence $E_n (\mathbf{w})$ is continuously
    differentiable.
    
  \item[b] If $y_n \mathbf{w}^T \mathbf{x}_n > 0$ this is trivial. If
    not, $(1 - y_n \mathbf{w}^T \mathbf{x}_n)^2 > 1$ and we're
    done. The inequality for the classification error now follows
    easily.

  \item[c] If we perform stochastic gradient descent with learning
    rate $\eta / 2$ (it's fine to just derive it with $\eta$, this
    version just ends up giving you back Adaline), then the update for
    a random data point $(\mathbf{x}_n, y_n)$ is given by $\mathbf{w}
    = \mathbf{w} - \frac{\eta}{2} \nabla E_n (\mathbf{w})$. Using the
    gradients derived in part (a), we get the updates:
    \[ \mathbf{w} = 
      \begin{cases}
        \mathbf{w} + \eta(1-y_n \mathbf{w}^T \mathbf{x}_n)y_n
        \mathbf{x}_n & \text{if } y_n \mathbf{w}^T \mathbf{x}_n < 1 \\
        \mathbf{w} & y_n \text{if } \mathbf{w}^T \mathbf{x}_n \geq 1
      \end{cases}
    \]

  \end{itemize}

\item[Problem 3.19] For part (a), as long as no two training examples
  are identical, this transform will always lead to a linearly
  separable dataset. The VC dimension is then infinity, so there's no
  generalization guarantee at all. For part (b), this feature
  transform is basically encoding similarity to the existing training
  examples, and as the number of training examples grows the
  complexity of the encoding grows with them, so again the VC
  dimension is infinite and you cannot guarantee generalization. By
  contrast, for part (c), the vector is large (101x101 + 1), so it
  won't work great for small $N$, but at least for very large $N$ one
  could eventually get some generalization.



\end{enumerate}


\end{document}
